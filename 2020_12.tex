\section{Exam Preparation 2020}
\subsection{Eigenvectors and eigenvalues}
Consider the matrix 
\begin{equation*}
    A = \begin{bmatrix}
        8 & 1 \\ 
        2 & 9
    \end{bmatrix}
\end{equation*}

\subsubsection{Find the eigenvalues and eigenvectors of \(A\).}

\subsubsection{Find matrices \(P\) and \(D\) such that \(D\) is diagonal and \(A=PDP^{-1}\)}

Consider a system given by a pair of some (unspecified numbers) \((x_0,y_0)\) and the rule: 
\begin{align*}
    x_{n+1} &= 0.8x_n + 0.1y_n \\ 
    y_{n+1} &= 0.2x_n + 0.9y_n
\end{align*}

\subsubsection{Construct a matrix \(B\) such that \[\begin{bmatrix}x_{n+1}\\y_{n+1}\end{bmatrix} = B \begin{bmatrix} x_n\\y_n\end{bmatrix}\]}


\subsection{Bases}
Consider the polynomials
\begin{align*}
    p(x) &= x^2 + 2\\ 
    q(x) &= 3x^2 + x - 1 \\ 
    r(x) &= 2x^2 + 4x + 1 
\end{align*}

\subsubsection{Show that \(B=\{p,q,r\}\) forms a basis for the vector space \(P_2\) of polynomials of degree at most 2}

\subsubsection{Compute the coordinates of the polynomial \(s(x)=5x^2+11x+8\) relative to the basis \(B\)}

\subsection{Determinants}
Consider the matrix
\begin{equation*}
    A = \begin{bmatrix}
        3 & -1 & 2 & 1\\ 
        -1 & 2 & 0 & -1 \\ 
        2 & -1 & 0 & 3 \\ 
        2 & -2 & 0 & 1
    \end{bmatrix}
\end{equation*}

\subsubsection{Compute the determinant of \(A\). Is \(A\) invertible?}

\subsection{Gradients and partial derivatives}
Consider the function \(F: \R^2 \rightarrow \R\) defined by
\begin{equation*}
    F(x,y) = 3xy^3 + x^2y^2-2xy
\end{equation*}

\subsubsection{Compute the gradient of \(F\) and compute the directional derivative of \(F\) at the point (2, 1) in the direction given by the unit vector \(u=\frac{1}{25}(-3,4)\)}

\subsubsection{ Compute a formula for the tangent line to the level curve given by \(F(x, y) = 6\) at the point (2, 1).Reduce as much as possible}
Suppose now that we are given a pair of functions \(g,h:\R \rightarrow\R\) satisfying the following properties 

\begin{align*}
    g(0) &= 2 \\ 
    h(0) &= 1 \\ 
    g'(0) &= -2 \\ 
    h'(0) &= 1 
\end{align*}

and consider the function \(k: \R \rightarrow \R \) defined as \(k(t) = F(g(t),h(t))\). 

\subsubsection{What is \(k'(0)\)?}

For the last question, consider the curve given by \(R: \R \rightarrow \R^2 \) defined as: 
\begin{equation*}
    R(t) = \left( \frac{3}{2}t^2,2t^2 \right) 
\end{equation*}

\subsubsection{Compute the arc length traversed by \(R\) from \(t=0\) to \(t=2\).}


\subsection{Optimisation problems}
\subsubsection{Use the method of Lagrange multipliers to compute the maximum of the function \(f(x, y) = x + y\) under the constraint that \(g(x,y)\leq 2\) where \(g(x,y) = (x-1)^2 + (y-2)^2\). }


\subsection{Multiple integrals}
\subsubsection{Compute the volume of the solid bounded by the graph of \(f(x,y)=6-2y+3x\), and \(z=0\) and over the region given by the restrictions \(0 \leq x \leq 4\) and \(0 \leq g \leq 2\).}


\subsubsection{Compute the volume of the solid bounded by the graph of \(f(x,y) = x^2y\), the plane \(z=0\) and over the region given by the restrictions \(y \geq 0\) and \(y + x^2 \leq 1\)}


