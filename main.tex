\documentclass[a4paper,11pt]{article}
\usepackage{times,fullpage,amsthm, amsmath, amsfonts,mathtools,systeme}
\usepackage[margin=1in]{geometry}
\usepackage{graphicx}
\setlength{\parindent}{0pt}
\newenvironment{amatrix}[1]{%
  \left[\begin{array}{@{}*{#1}{c}|c@{}}
}{%
  \end{array}\right]
}

\newenvironment{gjmatrix}[1]{%
  \left[\begin{array}{ccc | ccc}
}{%
  \end{array}\right]
}

\newcommand{\ro}[1]{%
  \xrightarrow{\mathmakebox[\rowidth]{#1}}%
}
\newlength{\rowidth}% row operation width
\AtBeginDocument{\setlength{\rowidth}{3em}}




\newcommand{\R}{\mathbb{R}}
\newcommand{\inv}[1]{{#1}^{-1}}


\title{Old Exams}
\date{\today}
\author{Data Science 2021}

\begin{document}

\maketitle

\section{Exam January 4 2018}

\subsection{Systems of Linear Equations (16 points)}
\subsubsection{Compute the set of solutions to the following system of linear equations [10 points]}

\begin{equation*}
    \sysdelim..\systeme{
    w + x + y - z = 3,
    2x + y = 4,
    -2w + x + z = 1
    }
  \end{equation*}

\subsubsection{Compute the rank of A and the dimensions the row space, column space and null space of A. [6 points]}
For this question consider the matrix: 

\begin{equation*}
    A = \begin{bmatrix}
        1 & 1 & 1 & -1 \\ 
        0 & 2 & 1 & 0 \\ 
        -2 & 1 & 0 & 1
    \end{bmatrix}
\end{equation*}


\subsection{Matrices (22 points)}
For this problem, consider the following 4 matrices: 

\begin{align*}
    A &= \begin{bmatrix}
        6 & 4 \\ 
        2 & 3 
    \end{bmatrix}\\ 
    B &= \begin{bmatrix}
        7 & 0 & 3 \\ 
        -9 & -2 & 3 \\ 
        18 & 0 & -8
    \end{bmatrix}\\ 
    C &= \begin{bmatrix}
        2 & -2 \\ 
        1 & 5 
    \end{bmatrix}\\ 
    v &= \begin{bmatrix}
        1 \\ 
        0 \\ 
        3
    \end{bmatrix}
\end{align*}

\subsubsection*{Compute the inverse of the matrix A.[5 points]}

\subsubsection*{Is the vector v an eigenvector for B? If so, what is the eigenvalue?[5 points]}

\subsubsection*{Compute the eigenvectors and eigenvalues of the matrix C.[12 points] }



\subsection{Vector Spaces (17 points)}
For the first question, consider the following three vectors: 
\begin{align*}
    u &= \begin{bmatrix}
        1 \\ 
        2 \\ 
        3
    \end{bmatrix} \\ 
    v &= \begin{bmatrix}
        2 \\ 
        3 \\ 
        5
    \end{bmatrix}\\ 
    w &= \begin{bmatrix}
        0 \\ 
        1 \\ 
        2
    \end{bmatrix}
\end{align*}

\subsubsection{ Are the vectors \(u, v, w\) linearly independent? Argue for your answer. [10 points]}

For the next question, consider the following two subsets of \(\R^2\): 

\begin{align*}
    V &= \{(x,y) | xy + y = 4 \} \\ 
    W &= \{(x,y) | 2x = 5y \} 
\end{align*}
\subsubsection{ Which of the two sets V, W are subspaces of \(\R^2\)? Argue for your answer.[7 points]}


\subsection{Gradients (15 points)}
For the questions below, consider the function \(f\) defined as
\begin{equation*}
    f(x,y) = 3xy^2-xe^y
\end{equation*}


\subsubsection{Compute the gradient of \(f\) [5 points]}


Suppose now that \(x,y : \R \rightarrow \R \) are differentiable functions satisfying 
\begin{align*}
    x(0) &= 2 \\ 
    y(0) &= 0 \\ 
    x'(0) &= 2 \\ 
    y'(0) &= 3
\end{align*}

Let \(g(t) = f(x(t),y(t))\) for \(f\) the function defined above. 

\subsubsection{Compute the derivative \(g'(0)\) [5 points]}


Suppose: \(\R^2 \rightarrow\R\)  is a function with the following two directional derivatives

\begin{align*}
    D_{u1}(h)(0,0) &= \sqrt{2}\\ 
    D_{u2}(h)(0,0) &= \frac{1}{\sqrt{2}} 
\end{align*}
for \(u_1,u_2\) the unit vectors 
\begin{align*}
    u_1 &= \left(\frac{1}{\sqrt{2}},\frac{1}{\sqrt{2}}\right)\\ 
    u_2 &= \left(\frac{1}{\sqrt{2}},\frac{-1}{\sqrt{2}}\right)\\ 
\end{align*}


\subsubsection{Compute the gradient \(\bigtriangledown h(0,0)\) (5 points)}


\subsection{Optimisation (10 points)}

\subsubsection{ Use the method of Langrange multipliers to compute the minimum of the function \(f(x,y)=x^2+y^2\) under the constraint \(g(x,y)=x^2y=16\) [10 points]} 


\subsection{Arc Lengths (8 points)}
Consider the curve in \(\R^2\) given by: 

\begin{align*}
  x(t) &= e^t\cos(t)\\ 
  y(t) &= e^t\sin(t)
\end{align*}
\subsubsection{Compute the arc length traversed by the curve from \(t=0\) to \(t=3\) [8 points]}

\subsection{Multiple Integrals (12 points)}
or this problem, consider the region \(D\) in the \(xy\)-plane bounded by the curves \(y=x^2\) and \(y=x\) and the lines \(x=0\) and \(x=1\). 

\subsubsection{Sketch the region \(D\) (2 points)}

\subsubsection{Compute the volume of the solid bounded by the graph of the function \(f(x,y)=xy^2\), the plane \(z=0\), and over the region \(D\)}\newpage 
\section{Exam March 6 2019}
\subsection{Systems of linear equations (15 points)}
\subsubsection{Compute the set of solutions to the following system of linear equations. (10 points)}
\begin{equation*}
    \sysdelim..\systeme{
    -2x_0 - 4x_1 + 2x_2 + 4x_3 = -8,
    x_0 + 2x_1 - 2x_2 + 4x_3 = 6,
    -2x_0 - 4x_1 + 3x_2 - 2x_3 = -10
    }
\end{equation*}

For the next question, consider the matrix

\begin{equation*}
    A = \begin{bmatrix}
        -2 &  -4 & 2 & 4\\ 
        1 & 2 & -2 & 4 \\ 
        -2 & -4 & 3 & -2
    \end{bmatrix}
\end{equation*}


\subsubsection{Compute a basis for the null space of \(A\) (5 points)}


\subsection{Determinants (5 points)}
Consider the matrix 
\begin{equation*}
    A = \begin{bmatrix}
        3 & 5 & 2 \\ 
        1 & -1 & 3 \\ 
        2 & 1 & -1 
    \end{bmatrix}
\end{equation*}

\subsubsection{Compute the determinant \(A\) (5 points)}


\subsection{Projections and bases (23 points)}
For this problem, consider the vectors

\begin{align*}
    u &= \begin{bmatrix}
        1 \\ -1 \\ 1
    \end{bmatrix}\\ 
    v &= \begin{bmatrix}
        1 \\ 0 \\ 2 
    \end{bmatrix}\\
    w &= \begin{bmatrix}
        1 \\ 2 \\ 3
    \end{bmatrix} \\ 
    z &= \begin{bmatrix}
        1 \\ 3 \\ 5
    \end{bmatrix}
\end{align*}

Let \(V\) be the subspace of \(\R^3\) spanned by \(u,v\). 

\subsubsection{Is \(w\) in \(V\)? Argue for your answer.(6 points)}


\subsubsection{Compute the coordinates \([z]_B\) of \(Z\) relative to the basis \(B={u,v}\) for \(V\) (7 points)}

\subsubsection{Compute the projection matrix from \(\R^3\) to \(V\).}


\subsection{PageRank (10 points)}
Consider the web consisting of 4 pages with links as indicated in the below diagram.

{\centering
\includegraphics[width=0.3\textwidth]{img/2019-03-06.png}
}
\subsubsection{Construct the matrix \(M\) for which the page ranking for the above web is an eigenvector for eigenvalue 1. The matrix should take dangling nodes into account and should use damping factor 0. Note that you are \textit{not} asked to compute the eigenvector of \(M\)}

\subsection{Gradients (17 points)}
For the questions below, consider the function \(f\) defined as
\begin{equation*}
    f(x,y) = (x-2)^3y^2
\end{equation*}
\subsubsection{Compute the gradient of \(f\) (5 points)}

\subsubsection{Construct a formula for the tangent to the level curve given by \(f(x,y) = 4\) at the point \((3,2)\) (6 points)}


\subsubsection{Construct a formular for the tangent plane to the graph of \(f\) above the point (0,1) (6 points)}


\subsection{Optimization (10 points)}
\subsubsection{Consider the function \(f(x,y)=2x^2+2xy-3y^2+x\). Find the critical points of \(f\) and determine for each one of them, if it is a local minimum, local maximum or a saddle point. (10 points)}

\subsection{Taylor polynomials (8 points)}
\subsubsection{Compute the 2nd order Taylor polynomial for the function \(f(x)=\cos(\pi x^2)\) at the point \(a=1\) (8 points)}


\subsection{Multiple integrals (12 points)}
For this problem, consider the region \(D\) in the \(xy\)-plane bounded by the curve \(y=x^2\) and the lines \(y=0\) and \(x=2\). 

\subsubsection{Sketch the region \(D\) (2 points)}

\subsubsection{Compute the volume of the solid bounded by the graph of the function \(f(x,y)=xe^y\), the plane \(z=0\), and over the region \(D\) (10 points)}
\newpage
\section{Exam January 6 2020}
\subsection{System of linear equations (15 points)}
\subsubsection{Compute the set of solutions to the following system of linear equations. (10 points)}
\begin{equation*}
    \sysdelim..\systeme{
    -2w + 2x + 2y - 8z = 14, 
    3w - 3x + y + 4z = -5,
    2w -2x -4y + 12z = -22 
    }
\end{equation*}

For the next question consider the matrix
\begin{equation*}
    A = \begin{bmatrix}
        -2 & 2 & 2 & -8 \\ 
        3 & -3 & 1 & 4 \\ 
        2 & -2 & -4 & 12
    \end{bmatrix}
\end{equation*}

\subsubsection{Compute the rank and the dimensions of the row and column space of \(A\). Give a basis for the null space of \(A\)}

\subsection{Eigenvectors and eigenvalues (20 points)}
For this problem consider the matrix
\begin{equation*}
    A = \begin{bmatrix}
        3 & -6 \\ 
        -5 & 2
    \end{bmatrix}
\end{equation*}

\subsubsection{Compute the eigenvectors and eigenvalues of the matrix \(A\) (12 points)}
Now consider also the vector 
\begin{equation*}
    v = \begin{bmatrix}
        3 \\ -8 
    \end{bmatrix}
\end{equation*}

\subsubsection{Write \(v\) as a linear combination of eigenvectors and use this to find numbers \(a,b,c,d\) and \(w,x,y,z\) such that:}
\begin{equation*}
    A^nv = \begin{bmatrix}
        a*b^n + c* d^n \\ 
        w*x^n + y * z^n
    \end{bmatrix}
\end{equation*}


\subsection{Vector spaces (10 points)}
For this problem, consider the vector space
\begin{equation*}
    V = \Bigg \{ \begin{bmatrix}
        a & b \\ 
        b & c
    \end{bmatrix}\Bigg | a,b,c \in \R \Bigg\} 
\end{equation*}

of symmetric 2 × 2 matrices. You may take for granted that this is a subspace of the vector space of 2 × 2 matrices discussed in the course, and do not have to prove this fact. Consider the three matrices

\begin{align*}
    A &= \begin{bmatrix}
        2 & 1 \\ 
        1 & 3 
    \end{bmatrix}\\ 
    B &= \begin{bmatrix}
        -1 & 2 \\ 
        2 & 1 
    \end{bmatrix}\\ 
    C &= \begin{bmatrix}
        0 & 1 \\ 
        1 & -2 
    \end{bmatrix}
\end{align*}

\subsubsection{Does the set \(S=\{A,B,C\}\) span the vector space \(V\)? Argue for your answer. (10 points)}


\subsection{Coefficients relative to a basis (6 points)}
For this problem consider the following three vectors in \(P_2\), the vector space of polynomials of degree
at most 2. 
\begin{align*}
    p &= x^2 + 2 \\ 
    q &= 1 -x \\ 
    r &= 3x^2 + 2x 
\end{align*}

You may take for granted that \(B = \{p,q,r\}\) forms a basis for \(P_2\) and do not have to prove this. 

\subsubsection{Compute the coordinates of the polynomial \(2x^2\) relative to \(B\) (6 points)}

\subsection{PageRank (9 points)}
Consider the web consisting of 4 pages with links as indicated in the below diagram. 

{\centering
\includegraphics[width=0.3\textwidth]{img/2020-01-06.png}
}


\subsubsection{Construct the matrix \(M\) for which the page ranking for the above web is an eigenvector for eigenvalue 1. The matrix should take dangling nodes into account and should use damping factor 0. Note that you are not asked to compute the eigenvector of \(M\). (7 points)}


For the next question, consider the following 3 vectors

\begin{align*}
    u &= \begin{bmatrix}
        2 \\ 4 \\ 1 \\ 4 
    \end{bmatrix}\\ 
    v &= \begin{bmatrix}
        1 \\ 3 \\ 2 \\ 3
    \end{bmatrix}\\ 
    w &= \begin{bmatrix}
        2 \\ 2 \\ 1 \\ 3
    \end{bmatrix}
\end{align*}

\subsubsection{Which of the three vectors \(u, v, w\) is an eigenvector for the matrix \(M\) constructed in the previous question? (2 points)}


\subsection{Gradients (18 points)}
For the questions below, consider the function \(f\) defined as 
\begin{equation*}
    f(x,y) = (x-2)^4e^{2y}
\end{equation*}

\subsubsection{Compute the gradient of \(f\) (6 points)}

\subsubsection{Let \(u = \left(\frac{3}{5},\frac{4}{5}\right) \) Compute the directional derivative \(D_u(f)(3,0)\) (6 points)} 

For the next question consider the functions \(x,y: \R \rightarrow \R\) defined as \(x(t) = \sin(2t), y(t)=e^t\) and suppose \(g: \R \rightarrow \R \) is a function satisfying \(\bigtriangledown g(0,1) = (3,-4)\). 

\subsubsection{Compute the derivative \(h'(0)\) for the function \(h\) defined as \(h(t)=g(x(t),y(t))\) (6 points) }


\subsection{Optimisation (10 points)}
\subsubsection{ Use the method of Langrange multipliers to compute the maximum and minimum of the function \(f(x,y)=(x-1)^2-y^2\) under the constraint \(g(x,y)=x^2+2y^2=1\) (10 points)}


\subsection{Multiple integrals (12 points)}
For this problem, consider the region \(D\) in the \(xy\)-plane bounded by the curves \(y=x^2\) and \(y=\sqrt{x}\) and the lines \(x=0\) and \(x=1\). 

\subsubsection{Sketch the region \(D\) (2 points)}


\subsubsection{Compute the volume of the solid bounded by the plane \(z=0\), the graph of the function \(f(x,y)=4xy+2y^3\) and over the region \(D\) (10 points)}
\newpage
\section{Exam April 24 2020}
\subsection{Systems of Linear Equations (15 points)}
\subsubsection{Compute the set of solutions to the following system of linear equations [10 points]}

\begin{equation*}
    \sysdelim..\systeme{
   -2w -4x -4y -16z = 10, 
   3w + 6x +4y +18z = -13,
   2w +4x +y + 9z = -9
    }
  \end{equation*}

For the next question consider the matrix

\begin{equation*}
    A = \begin{bmatrix}
        -2 & -4 & -4 & 16 \\ 
        3 & 6 & 4 & 18 \\ 
        2 & 4 & 1 & 9
    \end{bmatrix}
\end{equation*}

\subsubsection{Compute the rank and the dimensions of the row and column space of \(A\). Give a basis for the null space of \(A\). (5 points)}


\subsection{Eigenvectors and eigenvalues (18 points)}
For this problem consider the matrix
\begin{equation*}
    A = \begin{bmatrix}
        4 & 8 \\ 
        -1 & -2 
    \end{bmatrix}
\end{equation*}

\subsubsection{ Compute the eigenvectors and eigenvalues of the matrix \(A\). (12 points)}


\subsubsection{Find matrices \(P\) and \(D\) such that \(D\) is diagonal, \(P\) is invertible, and \(A = PDP^{-1}\) (6 points)}


\subsection{Vector spaces (12 points)}
For this problem, let \(A\) be the matrix
\begin{equation*}
    A = \begin{bmatrix}
        1 & 2 \\ 
        2 & 1
    \end{bmatrix}
\end{equation*}
and consider the subsets 

\begin{align*}
    V &= \Bigg \{ \begin{pmatrix}
        a & b \\ 
        c & d
    \end{pmatrix} \in M_{2,2} \Bigg | a = 0 \text{or} b=0 \Bigg \} \\ 
    W &= \{B \in M_{2,2} | AB = BA \} 
\end{align*}

of the set \(M_{2,2}\) of 2x2-matrices. 

\subsubsection{Which of the two subsets \(V\) and \(W\) are subspaces of \(M_{2,2}\)? Argue for your answer. (12 points)}

\subsection{Span and linear independence (15 points)}
For this problem, consider the four matrices
\begin{align*}
    A &= \begin{bmatrix}
        1 & 2 \\ 
        2 & -1
    \end{bmatrix}\\ 
    B &= \begin{bmatrix}
        2 & 3 \\ 
        -1 & 1
    \end{bmatrix}\\ 
    C &= \begin{bmatrix}
        4 & 5 \\ 
        -7 & 5 
    \end{bmatrix}\\ 
    D &= \begin{bmatrix}
        0 & 2 \\ 
        1 & -1 
    \end{bmatrix}
\end{align*}

These matrices will be considered as vectors in the vector space \(M_{2,2}\) of 2x2-matrices. 

\subsubsection{Is \(C\) in the subspace \(\text{span}\{A, B\}\)? Argue for your answer. (6 points)}


\subsubsection{Are the vectors \(A, B, D\) linearly independent? Argue for your answer. (9 points)}


\subsection{Gradients and tangent planes (12 points)}
For the questions below, consider the function \(f\) defined as
\begin{equation*}
    f(x,y)=e^{x^2}y
\end{equation*}
\subsubsection{Compute the gradient of \(f\) (6 points)}

\subsubsection{Give a formula for the tangent plane for \(f\) at (1, 1) (6 points)}

\subsection{Optimisation (8 points)}
\subsubsection{Consider the function \(f(x,y)=x^2+xy+2y^2+y-5x\). Find the critical points for \(f\)
and decide, for each one of them, if it is a local minimum, local maximum, or a saddle point.}


\subsection{Taylor polynomials (8 points)}
\subsubsection{Compute the 2nd order Taylor polynomial for the function \(f(x)=\sin(\frac{\pi}{2}e^x)\) around the point \(x=0\). (8 points)}


\subsection{Multiple integrals (12 points)}
For this problem, consider the region \(D\) in the \(xy\)-plane bounded by the curve \(y=x^2\) and the lines \(x=0\), \(x=\sqrt{\frac{\pi}{2}}\) and \(y=0\). Question \textbf{1} and \textbf{2} below are meant as preparation for solving question \textbf{3}

\subsubsection{Sketch the region \(D\) (1 points)}


\subsubsection{Compute the derivative of the function \(g(x)=\cos(x^2)\). (1 points)}

\subsubsection{Compute the volume of the solid bounded by the plane \(z = 0\), the graph of the function
\(f(x, y) = 2x \cos(y)\), and over the region \(D\).}\newpage
\section{Exam Preparation 2020}
\subsection{Eigenvectors and eigenvalues}
Consider the matrix 
\begin{equation*}
    A = \begin{bmatrix}
        8 & 1 \\ 
        2 & 9
    \end{bmatrix}
\end{equation*}

\subsubsection{Find the eigenvalues and eigenvectors of \(A\).}

\subsubsection{Find matrices \(P\) and \(D\) such that \(D\) is diagonal and \(A=PDP^{-1}\)}

Consider a system given by a pair of some (unspecified numbers) \((x_0,y_0)\) and the rule: 
\begin{align*}
    x_{n+1} &= 0.8x_n + 0.1y_n \\ 
    y_{n+1} &= 0.2x_n + 0.9y_n
\end{align*}

\subsubsection{Construct a matrix \(B\) such that \[\begin{bmatrix}x_{n+1}\\y_{n+1}\end{bmatrix} = B \begin{bmatrix} x_n\\y_n\end{bmatrix}\]}


\subsection{Bases}
Consider the polynomials
\begin{align*}
    p(x) &= x^2 + 2\\ 
    q(x) &= 3x^2 + x - 1 \\ 
    r(x) &= 2x^2 + 4x + 1 
\end{align*}

\subsubsection{Show that \(B=\{p,q,r\}\) forms a basis for the vector space \(P_2\) of polynomials of degree at most 2}

\subsubsection{Compute the coordinates of the polynomial \(s(x)=5x^2+11x+8\) relative to the basis \(B\)}

\subsection{Determinants}
Consider the matrix
\begin{equation*}
    A = \begin{bmatrix}
        3 & -1 & 2 & 1\\ 
        -1 & 2 & 0 & -1 \\ 
        2 & -1 & 0 & 3 \\ 
        2 & -2 & 0 & 1
    \end{bmatrix}
\end{equation*}

\subsubsection{Compute the determinant of \(A\). Is \(A\) invertible?}

\subsection{Gradients and partial derivatives}
Consider the function \(F: \R^2 \rightarrow \R\) defined by
\begin{equation*}
    F(x,y) = 3xy^3 + x^2y^2-2xy
\end{equation*}

\subsubsection{Compute the gradient of \(F\) and compute the directional derivative of \(F\) at the point (2, 1) in the direction given by the unit vector \(u=\frac{1}{25}(-3,4)\)}

\subsubsection{ Compute a formula for the tangent line to the level curve given by \(F(x, y) = 6\) at the point (2, 1).Reduce as much as possible}
Suppose now that we are given a pair of functions \(g,h:\R \rightarrow\R\) satisfying the following properties 

\begin{align*}
    g(0) &= 2 \\ 
    h(0) &= 1 \\ 
    g'(0) &= -2 \\ 
    h'(0) &= 1 
\end{align*}

and consider the function \(k: \R \rightarrow \R \) defined as \(k(t) = F(g(t),h(t))\). 

\subsubsection{What is \(k'(0)\)?}

For the last question, consider the curve given by \(R: \R \rightarrow \R^2 \) defined as: 
\begin{equation*}
    R(t) = \left( \frac{3}{2}t^2,2t^2 \right) 
\end{equation*}

\subsubsection{Compute the arc length traversed by \(R\) from \(t=0\) to \(t=2\).}


\subsection{Optimisation problems}
\subsubsection{Use the method of Lagrange multipliers to compute the maximum of the function \(f(x, y) = x + y\) under the constraint that \(g(x,y)\leq 2\) where \(g(x,y) = (x-1)^2 + (y-2)^2\). }


\subsection{Multiple integrals}
\subsubsection{Compute the volume of the solid bounded by the graph of \(f(x,y)=6-2y+3x\), and \(z=0\) and over the region given by the restrictions \(0 \leq x \leq 4\) and \(0 \leq g \leq 2\).}


\subsubsection{Compute the volume of the solid bounded by the graph of \(f(x,y) = x^2y\), the plane \(z=0\) and over the region given by the restrictions \(y \geq 0\) and \(y + x^2 \leq 1\)}


\newpage
\section{Exam January 29 2021}
\subsection{Systems of linear equations (15 points)}
\subsubsection{Compute the set of solutions to the following system of linear equations (10 points)}
\begin{equation*}
    \sysdelim..\systeme{
   3w + 6x + 6y + 15z = 21, 
   2w + 4x + 7y + 19z = 20, 
   3w + 6x + 4y + 9z = 17
    }
  \end{equation*}


For the next question consider the matrix
\begin{equation*}
    A = \begin{bmatrix}
        3 & 6 & 6 & 15 \\ 
        2 & 4 & 7 & 19 \\ 
        3 & 6 & 4 & 9
    \end{bmatrix}
\end{equation*}

\subsubsection{Compute the rank and the dimensions of the row and column space of \(A\). Give a basis for the null space of \(A\). (5 points)}


\subsection{Determinants (6 points)}
Consider the matrix
\begin{equation*}
    A = \begin{bmatrix}
        0 & 2 & 3 & 3 \\ 
        1 & -1 & 2 & -2 \\ 
        0 & 1 & 3 & 1 \\ 
        0 & 4 & 2 & -1 
    \end{bmatrix}
\end{equation*}

\subsubsection{Compute the determinant of \(A\). Is \(A\) invertible? (6 points)}


\subsection{Eigenvectors and eigenvalues (12 points)}
For this problem consider the matrix
\begin{equation*}
    A = \begin{bmatrix}
        -3 & -2 \\ 
        3 & 4
    \end{bmatrix}
\end{equation*}

\subsubsection{Compute the eigenvectors and eigenvalues of the matrix \(A\). (12 points)}

\subsection{Vector spaces (10 points)}
Let \(v\) be the vector
\begin{equation*}
    v = \begin{bmatrix}
        1 \\ 2
    \end{bmatrix}
\end{equation*}

and consider the subset
\begin{equation*}
    V = \{ A \in M_{2,2} | v^T A = (Av)^T \}
\end{equation*}

of the vector space \(M_{2,2}\) of 2x2-matrices. Here \(v^T\) is the transpose of \(v\). Consider also the matrix 
\begin{equation*}
    B = \begin{bmatrix}
        1 & 3 \\ 
        3 & 1
    \end{bmatrix}
\end{equation*}

\subsubsection{Is the matrix \(B\) an element in \(V\)? Argue for your answer (3 points)}

For the next question you may use the following equalities
\begin{align*}
    (D + E)^T &= D^T + E^T \\
    (cD)^T &= c(D^T) 
\end{align*}

You do not have to argue for these or prove them. These equalities hold for all matrices \(D, E\) and scalars \(c\) whenever \(D\) and \(E\) have the same dimensions. 


\subsubsection{Is \(V\) a subspace of \(M_{2,2}\)? Argue for your answer. (7 points)}


\subsection{Linear independence (10 points)}
Consider the following three polynomials
\begin{align*}
    p_1(x) &= x^3 -2x + 1 \\ 
    p_2(x) &= x^2 + 4x + 2 \\ 
    p_3(x) &= x^3 - x^2 + 1
\end{align*}

\subsubsection{Are the polynomials \(p_1,p_2 \text{and} p_3\)  linearly independent considered as vectors of the vector space \(P_3\) of polynomials of degree at most 3? Argue for your answer (10 points)}



\subsection{PageRank (7 points)}
Consider the web consisting of 5 pages with links as indicated in the below diagram. 

{\centering
\includegraphics[width=0.3\textwidth]{img/2021-01-28.png}
}

\subsubsection{ Construct the matrix \(M\) for which the page ranking for the above web is an eigenvector for eigenvalue 1. The matrix should take dangling nodes into account and should use damping factor 0. Note that you are not asked to compute the eigenvector of \(M\). (7 points)}




\subsection{Gradients and the chain rule (12 points)}
For the questions below, consider the function \(f\) defined as
\begin{equation*}
    f(x,y) = (x-y)^3
\end{equation*}

\subsubsection{Compute the gradient of \(f\) (6 points)}

For the next subquestion, suppose we are given functions \(g\) and \(h\) satisfying 
\begin{align*}
    g(0) &= 1 \\ 
    h(0) &= 0 \\ 
    g'(0) &= 2 \\ 
    h'(0) &= 1
\end{align*}
and let \(k\) be the function defined as \(k(t)=f(g(t),h(t))\), where \(f\) is as above. 

\subsubsection{What is the value of \(k'(0)\)? (6 points)}



\subsection{Taylor series (8 points)}
\subsubsection{Compute the 2nd order Taylor polynomial for the function \(f(x) = e^{x^2}\) around the basepoint 1. (8 points)}




\subsection{Optimisation (10 points)}
\subsubsection{Use Lagrange multipliers to find the pair \((x,y)\) with \(y > 0\) locally minimizing the function \(f(x,y)=x+4y\) under the constraint \(y^2+xy = 1 \). (10 points)}


\subsection{Double integrals (10 points)}
For this question, consider the region D in the plane bounded from below by the straight line from (0, 0) to (2, 1) and from above by the curve given by \(y = -x^2 +\frac{5}{2}x\) and between the lines \(x=0\) and \(x=2\). 

\subsubsection{Compute the value of the double integral \(\int\int_Df(x,y)dA\) where \(f(x,y)=x^2\)(10 points)}
\newpage
\section{Exam March 25 2021}
\subsection{ Systems of linear equations (15 points)}
\subsubsection{Compute the set of solutions to the following system of linear equations (10 points)}
\begin{equation*}
    \sysdelim..\systeme{
   5w + 10x -5y -15z = 100, 
   -2w -x -4y = -10, 
   3w + 4x + y - 8z = 52
    }
  \end{equation*}

For the next question consider the matrix 
\begin{equation*}
    A = \begin{bmatrix}
        5 & 10 & -5 & -15 \\
        -2 & -1 & -4 & 0 \\ 
        3 & 4 & 1 & -8
    \end{bmatrix}
\end{equation*}

\subsubsection{Compute the rank and the dimensions of the row and column space of \(A\). Give a basis for
the null space of \(A\). (5 points)}



\subsection{Eigenvectors and eigenvalues (26 points)}
For the first question, consider the matrix \(A\) and vector \(v\) defined as
\begin{align*}
    A &= \begin{bmatrix}
        1 & -2 & 3 \\ 
        -2 & 4 & -1 \\ 
        -1 & 2 & 7 
    \end{bmatrix}\\ 
    v &= \begin{bmatrix}
        2 \\ 1 \\ 0
    \end{bmatrix}
\end{align*}

\subsubsection{ Is \(v\) an eigenvector for the matrix \(A\)? Argue for your answer. (6 points)}


For the next question, consider the matrix B defined as
\begin{equation*}
    B = \begin{bmatrix}
        -2 & 2 \\ 
        -3 & 5
    \end{bmatrix}
\end{equation*}

\subsubsection{Compute the eigenvectors and eigenvalues of \(B\). (12 points)}


For the final question consider the vector
\begin{equation*}
    w = \begin{bmatrix}
        5 \\ 5
    \end{bmatrix}
\end{equation*}


\subsubsection{Write \(w\) as a linear combination of eigenvectors and use this to compute a closed formula
for \(B^nw\).}


\subsection{Projections (10 points)}
For this problem, consider the vectors
\begin{align*}
    u &= \begin{bmatrix}
        2 \\ 2 \\ 1
    \end{bmatrix}\\ 
    v &= \begin{bmatrix}
        1 \\ 1 \\ 2
    \end{bmatrix}\\ 
    w &= \begin{bmatrix}
        2 \\ 0 \\ 1
    \end{bmatrix}
\end{align*}
Let \(V\) be the subspace of \(\R^3\) spanned by \(u\) and \(v\). 

\subsubsection{Compute the projection matrix from \(\R^3\) to \(V\) and compute the projection of \(w\) onto \(V\) (10 points)}




\subsection{Coordinates (9 points)}
For this problem, consider the following 4 vectors in the vector space \(M_{2,2}\) of 2 × 2 matrices
\begin{align*}
    A &= \begin{bmatrix}
        2 & 0 \\ 
        1 & 1
    \end{bmatrix}\\ 
    B &= \begin{bmatrix}
        0 & 1 \\ 
        2 & 1
    \end{bmatrix}\\ 
    C &= \begin{bmatrix}
        1 & 1 \\ 
        3 & 0
    \end{bmatrix}\\ 
    D &= \begin{bmatrix}
        7 & 0 \\ 
        4 & 2
    \end{bmatrix}
\end{align*}


You may take for granted that the vectors \(A, B, C\) are linearly independent, and therefore form a basis for the subspace \(V\) spanned by them.

\subsubsection{Compute the coordinates of \(D\) for the vector relative to the basis consisting of \(A, B \text{and} C\) for the vector space \(V\). (9 points)}

\subsection{Gradients and tangent planes (12 points)}
Consider the function \(f(x,y) = \sin(e^{x^2}y) + 2\)

\subsubsection{Compute the gradient of \(f\). (6 points)}

\subsubsection{Give a formula for the tangent plane for \(f\) at the point \((x,y)=(1,\frac{2\pi}{e})\) (6 points)}




\subsection{Arc length (8 points)}
\subsubsection{Compute the arc length of the curve \(f(t)=(t^3,\frac{\sqrt{7}}{3}t^3)\) from \(t=0\) to \(t=1\). (8 points)}


\subsection{Optimisation (8 points)}
\subsubsection{Let \(f\) be the function defined as \(f(x,y)=x^2+2y^2+4xy+x-4y\). Compute the critical points for \(f\) and decide, for each one of them, if it is a local maximum, a local minimum or a saddle point. (8 points)}


\subsection{Double integrals (12 points)}
For the problem, let \(D\) be the triangle in the plane given by the corners (0, 0), (2, 1) and (0, 2).

\subsubsection{Sketch the region \(D\). (2 points)}


\subsubsection{Compute the double integral \(\int \int_D xydA\). (10 points)}
\newpage









A system of linear equations
\begin{equation*}
  \sysdelim..\systeme{
  -w+3x+4y-9z = 6,
  -2w+3x+2y-6z = -3,
  3w+x+8y-8z = 17
  }
\end{equation*}
%
A matrix
\[
 A  =  
\begin{bmatrix}
 -1 & 3 & -2 & 10  \\
 2 & -1 & 4 & -5  \\
 4 & 2 & 8 & 2  
\end{bmatrix}
\]
%
A reduction
\[
\begin{amatrix}{4}
  -1 & 3 & -2 & 10 & 9 \\
 2 & -1 & 4 & -5  & -8 \\
 4 & 2 & 8 & 2 & -8
\end{amatrix}
\to
\begin{amatrix}{4}
  1 & -3 & 2 & -10 & -9 \\
 0 & 5 & 0 & 15  & 10 \\
 0 & 14 & 0 & 42 & 28 
\end{amatrix}
\]
%
A sequence of reductions with operations on some arrows
\begin{align*}
\begin{amatrix}{4}
  -1 & 3 & -2 & 10 & 9 \\
 2 & -1 & 4 & -5  & -8 \\
 4 & 2 & 8 & 2 & -8
\end{amatrix}
& \ro{R_1\gets (-1)R_1}
\begin{amatrix}{4}
  1 & -3 & 2 & -10 & -9 \\
 2 & -1 & 4 & -5  & -8 \\
 4 & 2 & 8 & 2 & -8
\end{amatrix}
\\ 
& \ro{R_2 \gets R_2 - 2R_1}
\begin{amatrix}{4}
  1 & -3 & 2 & -10 & -9 \\
 0 & 5 & 0 & 15  & 10 \\
 4 & 2 & 8 & 2 & -8
\end{amatrix}
\\ 
& \ro{}
\begin{amatrix}{4}
  1 & -3 & 2 & -10 & -9 \\
 0 & 5 & 0 & 15  & 10 \\
 0 & 14 & 0 & 42 & 28
\end{amatrix}
\end{align*}

\section{Question 2}

An inverted matrix

\[
\inv{
\begin{bmatrix}
 a & b \\ 
 c & d
\end{bmatrix}
}
=
\frac 1{ad-bc}
\begin{bmatrix}
 d & -b \\ 
 -c & a
\end{bmatrix}
\]

\section{Question 3}

A matrix with a divider in the middle
\[
\begin{gjmatrix}{3}
 1 & 2 & 3 & 1 & 0 & 0 \\
 0 & 4 & 1 & 0 & 1 & 0 \\
 -1 & 0 & 2 & 0 & 0 & 1 
\end{gjmatrix}
\]

\end{document}

