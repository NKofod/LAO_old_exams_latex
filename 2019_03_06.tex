\section{Exam March 6 2019}
\subsection{Systems of linear equations (15 points)}
\subsubsection{Compute the set of solutions to the following system of linear equations. (10 points)}
\begin{equation*}
    \sysdelim..\systeme{
    -2x_0 - 4x_1 + 2x_2 + 4x_3 = -8,
    x_0 + 2x_1 - 2x_2 + 4x_3 = 6,
    -2x_0 - 4x_1 + 3x_2 - 2x_3 = -10
    }
\end{equation*}

For the next question, consider the matrix

\begin{equation*}
    A = \begin{bmatrix}
        -2 &  -4 & 2 & 4\\ 
        1 & 2 & -2 & 4 \\ 
        -2 & -4 & 3 & -2
    \end{bmatrix}
\end{equation*}


\subsubsection{Compute a basis for the null space of \(A\) (5 points)}


\subsection{Determinants (5 points)}
Consider the matrix 
\begin{equation*}
    A = \begin{bmatrix}
        3 & 5 & 2 \\ 
        1 & -1 & 3 \\ 
        2 & 1 & -1 
    \end{bmatrix}
\end{equation*}

\subsubsection{Compute the determinant \(A\) (5 points)}


\subsection{Projections and bases (23 points)}
For this problem, consider the vectors

\begin{align*}
    u &= \begin{bmatrix}
        1 \\ -1 \\ 1
    \end{bmatrix}\\ 
    v &= \begin{bmatrix}
        1 \\ 0 \\ 2 
    \end{bmatrix}\\
    w &= \begin{bmatrix}
        1 \\ 2 \\ 3
    \end{bmatrix} \\ 
    z &= \begin{bmatrix}
        1 \\ 3 \\ 5
    \end{bmatrix}
\end{align*}

Let \(V\) be the subspace of \(\R^3\) spanned by \(u,v\). 

\subsubsection{Is \(w\) in \(V\)? Argue for your answer.(6 points)}


\subsubsection{Compute the coordinates \([z]_B\) of \(Z\) relative to the basis \(B={u,v}\) for \(V\) (7 points)}

\subsubsection{Compute the projection matrix from \(\R^3\) to \(V\).}


\subsection{PageRank (10 points)}
Consider the web consisting of 4 pages with links as indicated in the below diagram.

{\centering
\includegraphics[width=0.3\textwidth]{img/2019-03-06.png}
}
\subsubsection{Construct the matrix \(M\) for which the page ranking for the above web is an eigenvector for eigenvalue 1. The matrix should take dangling nodes into account and should use damping factor 0. Note that you are \textit{not} asked to compute the eigenvector of \(M\)}

\subsection{Gradients (17 points)}
For the questions below, consider the function \(f\) defined as
\begin{equation*}
    f(x,y) = (x-2)^3y^2
\end{equation*}
\subsubsection{Compute the gradient of \(f\) (5 points)}

\subsubsection{Construct a formula for the tangent to the level curve given by \(f(x,y) = 4\) at the point \((3,2)\) (6 points)}


\subsubsection{Construct a formular for the tangent plane to the graph of \(f\) above the point (0,1) (6 points)}


\subsection{Optimization (10 points)}
\subsubsection{Consider the function \(f(x,y)=2x^2+2xy-3y^2+x\). Find the critical points of \(f\) and determine for each one of them, if it is a local minimum, local maximum or a saddle point. (10 points)}

\subsection{Taylor polynomials (8 points)}
\subsubsection{Compute the 2nd order Taylor polynomial for the function \(f(x)=\cos(\pi x^2)\) at the point \(a=1\) (8 points)}


\subsection{Multiple integrals (12 points)}
For this problem, consider the region \(D\) in the \(xy\)-plane bounded by the curve \(y=x^2\) and the lines \(y=0\) and \(x=2\). 

\subsubsection{Sketch the region \(D\) (2 points)}

\subsubsection{Compute the volume of the solid bounded by the graph of the function \(f(x,y)=xe^y\), the plane \(z=0\), and over the region \(D\) (10 points)}
