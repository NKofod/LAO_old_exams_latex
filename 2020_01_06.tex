\section{Exam January 6 2020}
\subsection{System of linear equations (15 points)}
\subsubsection{Compute the set of solutions to the following system of linear equations. (10 points)}
\begin{equation*}
    \sysdelim..\systeme{
    -2w + 2x + 2y - 8z = 14, 
    3w - 3x + y + 4z = -5,
    2w -2x -4y + 12z = -22 
    }
\end{equation*}

For the next question consider the matrix
\begin{equation*}
    A = \begin{bmatrix}
        -2 & 2 & 2 & -8 \\ 
        3 & -3 & 1 & 4 \\ 
        2 & -2 & -4 & 12
    \end{bmatrix}
\end{equation*}

\subsubsection{Compute the rank and the dimensions of the row and column space of \(A\). Give a basis for the null space of \(A\)}

\subsection{Eigenvectors and eigenvalues (20 points)}
For this problem consider the matrix
\begin{equation*}
    A = \begin{bmatrix}
        3 & -6 \\ 
        -5 & 2
    \end{bmatrix}
\end{equation*}

\subsubsection{Compute the eigenvectors and eigenvalues of the matrix \(A\) (12 points)}
Now consider also the vector 
\begin{equation*}
    v = \begin{bmatrix}
        3 \\ -8 
    \end{bmatrix}
\end{equation*}

\subsubsection{Write \(v\) as a linear combination of eigenvectors and use this to find numbers \(a,b,c,d\) and \(w,x,y,z\) such that:}
\begin{equation*}
    A^nv = \begin{bmatrix}
        a*b^n + c* d^n \\ 
        w*x^n + y * z^n
    \end{bmatrix}
\end{equation*}


\subsection{Vector spaces (10 points)}
For this problem, consider the vector space
\begin{equation*}
    V = \Bigg \{ \begin{bmatrix}
        a & b \\ 
        b & c
    \end{bmatrix}\Bigg | a,b,c \in \R \Bigg\} 
\end{equation*}

of symmetric 2 × 2 matrices. You may take for granted that this is a subspace of the vector space of 2 × 2 matrices discussed in the course, and do not have to prove this fact. Consider the three matrices

\begin{align*}
    A &= \begin{bmatrix}
        2 & 1 \\ 
        1 & 3 
    \end{bmatrix}\\ 
    B &= \begin{bmatrix}
        -1 & 2 \\ 
        2 & 1 
    \end{bmatrix}\\ 
    C &= \begin{bmatrix}
        0 & 1 \\ 
        1 & -2 
    \end{bmatrix}
\end{align*}

\subsubsection{Does the set \(S=\{A,B,C\}\) span the vector space \(V\)? Argue for your answer. (10 points)}


\subsection{Coefficients relative to a basis (6 points)}
For this problem consider the following three vectors in \(P_2\), the vector space of polynomials of degree
at most 2. 
\begin{align*}
    p &= x^2 + 2 \\ 
    q &= 1 -x \\ 
    r &= 3x^2 + 2x 
\end{align*}

You may take for granted that \(B = \{p,q,r\}\) forms a basis for \(P_2\) and do not have to prove this. 

\subsubsection{Compute the coordinates of the polynomial \(2x^2\) relative to \(B\) (6 points)}

\subsection{PageRank (9 points)}
Consider the web consisting of 4 pages with links as indicated in the below diagram. 

{\centering
\includegraphics[width=0.3\textwidth]{img/2020-01-06.png}
}


\subsubsection{Construct the matrix \(M\) for which the page ranking for the above web is an eigenvector for eigenvalue 1. The matrix should take dangling nodes into account and should use damping factor 0. Note that you are not asked to compute the eigenvector of \(M\). (7 points)}


For the next question, consider the following 3 vectors

\begin{align*}
    u &= \begin{bmatrix}
        2 \\ 4 \\ 1 \\ 4 
    \end{bmatrix}\\ 
    v &= \begin{bmatrix}
        1 \\ 3 \\ 2 \\ 3
    \end{bmatrix}\\ 
    w &= \begin{bmatrix}
        2 \\ 2 \\ 1 \\ 3
    \end{bmatrix}
\end{align*}

\subsubsection{Which of the three vectors \(u, v, w\) is an eigenvector for the matrix \(M\) constructed in the previous question? (2 points)}


\subsection{Gradients (18 points)}
For the questions below, consider the function \(f\) defined as 
\begin{equation*}
    f(x,y) = (x-2)^4e^{2y}
\end{equation*}

\subsubsection{Compute the gradient of \(f\) (6 points)}

\subsubsection{Let \(u = \left(\frac{3}{5},\frac{4}{5}\right) \) Compute the directional derivative \(D_u(f)(3,0)\) (6 points)} 

For the next question consider the functions \(x,y: \R \rightarrow \R\) defined as \(x(t) = \sin(2t), y(t)=e^t\) and suppose \(g: \R \rightarrow \R \) is a function satisfying \(\bigtriangledown g(0,1) = (3,-4)\). 

\subsubsection{Compute the derivative \(h'(0)\) for the function \(h\) defined as \(h(t)=g(x(t),y(t))\) (6 points) }


\subsection{Optimisation (10 points)}
\subsubsection{ Use the method of Langrange multipliers to compute the maximum and minimum of the function \(f(x,y)=(x-1)^2-y^2\) under the constraint \(g(x,y)=x^2+2y^2=1\) (10 points)}


\subsection{Multiple integrals (12 points)}
For this problem, consider the region \(D\) in the \(xy\)-plane bounded by the curves \(y=x^2\) and \(y=\sqrt{x}\) and the lines \(x=0\) and \(x=1\). 

\subsubsection{Sketch the region \(D\) (2 points)}


\subsubsection{Compute the volume of the solid bounded by the plane \(z=0\), the graph of the function \(f(x,y)=4xy+2y^3\) and over the region \(D\) (10 points)}
