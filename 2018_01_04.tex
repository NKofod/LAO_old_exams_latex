\section{Exam January 4 2018}

\subsection{Systems of Linear Equations (16 points)}
\subsubsection{Compute the set of solutions to the following system of linear equations [10 points]}

\begin{equation*}
    \sysdelim..\systeme{
    w + x + y - z = 3,
    2x + y = 4,
    -2w + x + z = 1
    }
  \end{equation*}

\subsubsection{Compute the rank of A and the dimensions the row space, column space and null space of A. [6 points]}
For this question consider the matrix: 

\begin{equation*}
    A = \begin{bmatrix}
        1 & 1 & 1 & -1 \\ 
        0 & 2 & 1 & 0 \\ 
        -2 & 1 & 0 & 1
    \end{bmatrix}
\end{equation*}


\subsection{Matrices (22 points)}
For this problem, consider the following 4 matrices: 

\begin{align*}
    A &= \begin{bmatrix}
        6 & 4 \\ 
        2 & 3 
    \end{bmatrix}\\ 
    B &= \begin{bmatrix}
        7 & 0 & 3 \\ 
        -9 & -2 & 3 \\ 
        18 & 0 & -8
    \end{bmatrix}\\ 
    C &= \begin{bmatrix}
        2 & -2 \\ 
        1 & 5 
    \end{bmatrix}\\ 
    v &= \begin{bmatrix}
        1 \\ 
        0 \\ 
        3
    \end{bmatrix}
\end{align*}

\subsubsection*{Compute the inverse of the matrix A.[5 points]}

\subsubsection*{Is the vector v an eigenvector for B? If so, what is the eigenvalue?[5 points]}

\subsubsection*{Compute the eigenvectors and eigenvalues of the matrix C.[12 points] }



\subsection{Vector Spaces (17 points)}
For the first question, consider the following three vectors: 
\begin{align*}
    u &= \begin{bmatrix}
        1 \\ 
        2 \\ 
        3
    \end{bmatrix} \\ 
    v &= \begin{bmatrix}
        2 \\ 
        3 \\ 
        5
    \end{bmatrix}\\ 
    w &= \begin{bmatrix}
        0 \\ 
        1 \\ 
        2
    \end{bmatrix}
\end{align*}

\subsubsection{ Are the vectors \(u, v, w\) linearly independent? Argue for your answer. [10 points]}

For the next question, consider the following two subsets of \(\R^2\): 

\begin{align*}
    V &= \{(x,y) | xy + y = 4 \} \\ 
    W &= \{(x,y) | 2x = 5y \} 
\end{align*}
\subsubsection{ Which of the two sets V, W are subspaces of \(\R^2\)? Argue for your answer.[7 points]}


\subsection{Gradients (15 points)}
For the questions below, consider the function \(f\) defined as
\begin{equation*}
    f(x,y) = 3xy^2-xe^y
\end{equation*}


\subsubsection{Compute the gradient of \(f\) [5 points]}


Suppose now that \(x,y : \R \rightarrow \R \) are differentiable functions satisfying 
\begin{align*}
    x(0) &= 2 \\ 
    y(0) &= 0 \\ 
    x'(0) &= 2 \\ 
    y'(0) &= 3
\end{align*}

Let \(g(t) = f(x(t),y(t))\) for \(f\) the function defined above. 

\subsubsection{Compute the derivative \(g'(0)\) [5 points]}


Suppose: \(\R^2 \rightarrow\R\)  is a function with the following two directional derivatives

\begin{align*}
    D_{u1}(h)(0,0) &= \sqrt{2}\\ 
    D_{u2}(h)(0,0) &= \frac{1}{\sqrt{2}} 
\end{align*}
for \(u_1,u_2\) the unit vectors 
\begin{align*}
    u_1 &= \left(\frac{1}{\sqrt{2}},\frac{1}{\sqrt{2}}\right)\\ 
    u_2 &= \left(\frac{1}{\sqrt{2}},\frac{-1}{\sqrt{2}}\right)\\ 
\end{align*}


\subsubsection{Compute the gradient \(\bigtriangledown h(0,0)\) (5 points)}


\subsection{Optimisation (10 points)}

\subsubsection{ Use the method of Langrange multipliers to compute the minimum of the function \(f(x,y)=x^2+y^2\) under the constraint \(g(x,y)=x^2y=16\) [10 points]} 


\subsection{Arc Lengths (8 points)}
Consider the curve in \(\R^2\) given by: 

\begin{align*}
  x(t) &= e^t\cos(t)\\ 
  y(t) &= e^t\sin(t)
\end{align*}
\subsubsection{Compute the arc length traversed by the curve from \(t=0\) to \(t=3\) [8 points]}

\subsection{Multiple Integrals (12 points)}
or this problem, consider the region \(D\) in the \(xy\)-plane bounded by the curves \(y=x^2\) and \(y=x\) and the lines \(x=0\) and \(x=1\). 

\subsubsection{Sketch the region \(D\) (2 points)}

\subsubsection{Compute the volume of the solid bounded by the graph of the function \(f(x,y)=xy^2\), the plane \(z=0\), and over the region \(D\)}